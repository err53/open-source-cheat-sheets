% IF YOU CAN SEE THIS GO CONTRIBUTE >:(

\documentclass[letterpaper, 8pt]{extarticle}
\usepackage{amssymb,amsmath,amsthm,amsfonts}
\usepackage{multicol,multirow}
\usepackage{calc}
\usepackage{ifthen}
\usepackage[landscape]{geometry}
\usepackage[colorlinks=true,citecolor=blue,linkcolor=blue]{hyperref}

\usepackage{booktabs}
\usepackage{ulem}
\usepackage{enumitem}
\usepackage{tabulary}
\usepackage{graphicx}
\usepackage{siunitx}
\usepackage{tikz}
\usepackage{derivative}
\usepackage{svg}
\usepackage{listings}
\usepackage{setspace}
\usepackage{listings}
\usepackage{xcolor}
\usepackage{courier}
\usepackage{syntax}

% minimal line spacing
\setstretch{0.1}

% set borders (experimentally determined to minimize cutoff and maximize space on school printers)
\geometry{top=.25in,left=.25in,right=.25in,bottom=.35in}

% make figures work better in multicol
\newenvironment{Figure}
{\par\medskip\noindent\minipage}
{\endminipage\par\medskip}

\pagestyle{empty} % clear page

% rewrite section commands to be smaller
\makeatletter
\renewcommand{\section}{\@startsection{section}{1}{0mm}%
                                {-1explus -.5ex minus -.2ex}%
                                {0.5ex plus .2ex}%x
                                {\normalfont\normalsize\bfseries}}
\renewcommand{\subsection}{\@startsection{subsection}{2}{0mm}%
                                {-1explus -.5ex minus -.2ex}%
                                {0.5ex plus .2ex}%
                                {\normalfont\small\bfseries}}
\renewcommand{\subsubsection}{\@startsection{subsubsection}{3}{0mm}%
                                {-1ex plus -.5ex minus -.2ex}%
                                {1ex plus .2ex}%
                                {\normalfont\tiny\bfseries}}
\makeatother
\setcounter{secnumdepth}{0} % disable section numbering

% disable indenting
\setlength{\parindent}{0pt}
\setlength{\parskip}{0pt plus 0.5ex}

% Custom siunitx defs
\DeclareSIUnit\noop{\relax}
\NewDocumentCommand\prefixvalue{m}{%
\qty[prefix-mode=extract-exponent,print-unity-mantissa=false]{1}{#1\noop}
}

% Shorthand definitions
\newcommand{\To}{\Rightarrow}
\newcommand{\ttt}{\texttt}

% condense itemize & enumerate
\let\olditemize=\itemize \let\endolditemize=\enditemize \renewenvironment{itemize}{\olditemize \itemsep0em}{\endolditemize}
\let\oldenumerate=\enumerate \let\endoldenumerate=\endenumerate \renewenvironment{enumerate}{\oldenumerate \itemsep0em}{\endoldenumerate}
\setlist[itemize]{noitemsep, topsep=0pt, leftmargin=*}
\setlist[enumerate]{noitemsep, topsep=0pt, leftmargin=*}

\title{3MI3}

\begin{document}
\raggedright
\tiny

% make listings look nicer
\lstset{
    tabsize = 2, %% set tab space width
    showstringspaces = false, %% prevent space marking in strings, string is defined as the text that is generally printed directly to the console
    basicstyle = \tiny\ttfamily, %% set listing font and size
    breaklines = true, %% enable line breaking
    numberstyle = \tiny,
    postbreak = \mbox{\textcolor{red}{\(\hookrightarrow\)}\space}
}

\begin{center}
    {\textbf{3MI3 Final -- Year of the Rabbit Edition}} \\
\end{center}
% set column spacing rules
\setlength{\premulticols}{1pt}
\setlength{\postmulticols}{1pt}
\setlength{\multicolsep}{1pt}
\setlength{\columnsep}{2pt}
\begin{multicols*}{6}
    \section{Syntax}
    How a programming language ``looks''.
    Often a string, but can be a picture (Monet), or a grid of cells (Excel).
    \subsection{BNF}
    Formal specification of string-based syntaxes.
    \begin{grammar}
        <e> ::= x
        \alt \(\lambda x <e>\)
        \alt <e> <e>
        \alt (<e>)
    \end{grammar}
    \section{Semantics}
    \subsection{Dynamic Semantics}
    % Small-step semantics
    % -> Lambda calculus - Call-by-name vs Call-by-value
    % domain-specific languages
    % shallow vs deep vs tagless embeddings
    \subsection{Static Semantics}
    % types
    % Progress rule (always be able to make a step towards the value)
    % (for every well-typed program, either I can make a step, or e is a value)
    % Preservation rule ()

    % context list and how it works
    % unification (????? idk i didnt pay attention to this section)
    % u jus like me fr

    % featherweight java (wtf?)

    \section{Evaluation Strategies}

    Can't evaluate anything under a lambda in both of the below strategies.

    \subsection{Call by name}
    Can't evaluate anything that isn't an outermost term.
    % idk put the rules or an example here

    \subsection{Call by Value}
    Reduce arguments first.

    \section{Q1 Practice Answer}
    % CBN Example
    %((λx.λy.y y)((λw.w w)(λw.w w)))(λw.w)  
    %(λy.y y) (λw.w) - λx is called, all x’s replaced w/ (λw.w w)(λw.w w) 
    %(λw.w) (λw.w) - λy is called, all y’s replaced w/ (λw.w)
    %(λw.w) - λw is called, all w’s replaced w/ (λw.w)

    %CBV Example
    % (λx.λy.y y)((λw.w w)(λw.w w))(λw.w) 
    %(λx.λy.y y)((λw.w w)(λw.w w))(λw.w) - λw is called, all w’s replaced w/ (λw.w w)
    % Does not terminate, infinite loop of (λw.w w)(λw.w w)

    \section{Subtyping}
    T-Sub: \(\frac{\Gamma \vdash t : S \quad S <: T}{\Gamma \vdash t : T}\)\\
    S-RcdWidth: \(\{ l_{i}: T_{i}^{i \in  1..n +k} \} <: \{ l_{i}: T_{i}^{i \in  1..n} \}\)
    S-RcdDepth: \(\frac{\text{for each } i \quad S_{i} <: T_{i}}{\{ l_{i}: S_{i}^{i \in  1..n } \} <: \{ l_{i}: T_{i}^{i \in  1..n} \}}\)
    S-RcdPerm: \(\frac{\{k_{j} :S_{j}^{j \in 1..n}\} \text{ is perm. of } \{l_{i}:T_{i}^{i \in 1..n} \}}{\{k_{j}:S_{j}^{j \in 1..n} \} <: \{l_{i}: T_{I}^{i \in 1..n} \}}\)
    S-Arrow: \(\frac{T_{1} <: S_{1} \quad S_{2} <: T_{2}}{S_{1} \to S_{2} <: T_{1} \to T_{2}}\)\\
    This S-Arrow subtype relation is \textbf{contravariant} in the left-hand sides of arrows, and
    \textbf{covariant} in the right-hand sides of arrows.
    The intuition is that if we have a function \(S_1 \to S_2\),
    then it will accept any subtype of \(S_1\) as input, 
    and the result \(S_2\) can be viewed as belonging to any supertype of \(S_2\).

\end{multicols*}

\end{document}
